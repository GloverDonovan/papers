\documentclass{article}

% Packages ======================================

\usepackage{microtype}  % Make text spacing look a lot better
\usepackage{amsthm}     % "Theorem", "Definition", and other blocks
\usepackage{kpfonts}    % The font of choice
\usepackage{hyperref}   % Link support

% Package-specific settings =====================

\hypersetup{colorlinks=true}
\theoremstyle{definition}
\newtheorem{definition}{Definition}

% Document settings =============================

\title{Mica: A Specification}
\author{Donovan Glover}
\date{\today}

\begin{document}

    % ===========================================
    % Title Page & Table of Contents
    % ===========================================

	\maketitle
    \tableofcontents

    \newpage

    % ===========================================
    % Overview
    % ===========================================

    \section{Overview}

    \textbf{Mica} is \emph{a universal way to change color schemes.} It is written in the \href{https://crystal-lang.org}{Crystal programming language} and strives to \textbf{do one thing and do it well:} change color schemes.

    \newpage

    % ===========================================
    % Definitions
    % ===========================================

    \section{Definitions}

    \begin{definition}
        We define a \emph{color scheme} to be a set of 16 colors used as part of the operating system.
    \end{definition}

\end{document}
